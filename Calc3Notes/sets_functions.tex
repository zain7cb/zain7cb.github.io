\documentclass{article}

\usepackage{amsmath, amsthm, amssymb, amsfonts}
\usepackage{thmtools}
\usepackage{graphicx}
\usepackage{setspace}
\usepackage{geometry}
\usepackage{float}
\usepackage{hyperref}
\usepackage[utf8]{inputenc}
\usepackage[english]{babel}
\usepackage{framed}
\usepackage[dvipsnames]{xcolor}
\usepackage{tcolorbox}

\colorlet{LightGray}{White!90!Periwinkle}
\colorlet{LightOrange}{Orange!15}
\colorlet{LightGreen}{Green!15}

\newcommand{\HRule}[1]{\rule{\linewidth}{#1}}

\declaretheoremstyle[name=Theorem,]{thmsty}
\declaretheorem[style=thmsty,numberwithin=section]{theorem}
\tcolorboxenvironment{theorem}{colback=LightGray}

\declaretheoremstyle[name=Proposition,]{prosty}
\declaretheorem[style=prosty,numberlike=theorem]{proposition}
\tcolorboxenvironment{proposition}{colback=LightOrange}

\declaretheoremstyle[name=Principle,]{prcpsty}
\declaretheorem[style=prcpsty,numberlike=theorem]{principle}
\tcolorboxenvironment{principle}{colback=LightGreen}

\setstretch{1.2}
\geometry{
    textheight=9in,
    textwidth=5.5in,
    top=1in,
    headheight=12pt,
    headsep=25pt,
    footskip=30pt
}

% ------------------------------------------------------------------------------

\begin{document}

% ------------------------------------------------------------------------------
% Cover Page and ToC
% ------------------------------------------------------------------------------


% ------------------------------------------------------------------------------

\section{Sets}

We have no need to dive deeply into set theory, but I believe it will be helpful for us to have some of its tools and language. 
A \textbf{set} is really just a collection of objects. For example,
we can define $\mathbb{Z}$ to be the set of all integers
\[\mathbb{Z}=\{\ldots,-2,-1,0,1,2,\ldots\}.\]
An object $x$ inside a set $A$ is called an \textbf{element} of the set (or a \textbf{member} or \textbf{point}), and this relation is denoted $x \in A$ (which one can read as ``$x$ belongs to $A$'' or ``$x$ is in $A$'').
If $x$ does not belong to $A$, we write $x \notin A$.

We say a set $A$ is a \textbf{subset} of another set $B$ if every element that appears in $A$ also appears in $B$, and we write this relation as 
$A \subset B$. In other words, $A \subset B$ if whenever we have $x \in A$, then we also have $x \in B$.
For example, $\mathbb{Z}^+ = \{1,2,3,\ldots\}$ is a subset of $\mathbb{Z}$. Note that any set is a subset of itself.

Some sets are too large for us to describe them by simply listing out the elements and putting them between curly brackets. The real numbers $\mathbb{R}$ are an example of this.
The set of real numbers is still a perfectly fine set, however.  

Quite often, it is natural to describe a set $A$ as a subset of another set $B$ by picking out those elements of $B$ that meet a certain condition.
For example, we can describe the positive integers as 
\[ \{ x \in \mathbb{Z} : x > 0\}, \]
which one reads as ``all $x$ in $\mathbb{Z}$ such that $x > 0$''.
As another example, consider
\[ A = \{ x \in \mathbb{R} : |x| > 2\}.\]
That is, $A$ is defined to be those real numbers that have absolute value greater than $2$. In interval notation, $A = (-\infty, -2) \cup (2, \infty)$.
In general, this type of \textbf{set-builder} notation can be thought of as some kind vetting process, where we look at all the elements of some
given set, keeping the ones satisfying the given condition and tossing out the ones that don't.

There are some basic operations one can perform with sets. Let $A$ and $B$ be sets that are subsets of some larger set $U$. Then the \textbf{union} of these two sets, denoted $A \cup B$,
is defined to be the set of all elements that appear in at least one of $A$ or $B$. That is,
\[ A \cup B = \{x \in U : x \in A \text{ or } x \in B\}. \]
Note that we use ``or'' in the inclusive sense, meaning the condition $x \in A \text{ or } x \in B$ is satisfied so long as $x$ is in at least one of $A,B$, so
we allow the possibility of $x$ being in both.
The \textbf{intersection} of $A$ and $B$, denoted $A \cap B$, is defined as 
\[A \cap B = \{x \in U: x \in A \text{ and } x \in B\}.\]
The \textbf{complement} of $A$ (in $U$), denoted $A^C$, is defined as
\[A^C = \{x \in U: x \notin A\}.\]

Another useful notion is that of the \textbf{Cartesian product}, denoted $A \times B$, of two sets $A$ and $B$. This is defined as a collection of ordered pairs,
where the first ``coordinate'' is an element of $A$ and the second an element of $B$:
\[A \times B = \{(a,b): a \in A \text{ and } b \in B\}.\]
A familiar example is $\mathbb{R} \times \mathbb{R}$, the Cartesian plane. $\mathbb{R} \times \mathbb{R} \times \mathbb{R}$, usually 
abbreviated as $\mathbb{R}^3$, is the familiar space of three dimensions.

\section{Functions}

A function is an object that specifies a way of assigning elements of one set to elements of another set, and for our purposes,
this informal description is essentially enough. When we write $f: A \rightarrow B$, this is to say
$f$ is a function that takes in a point of $A$ and spits out a unique point in $B$ (i.e.\ a function cannot spit out two different values for one input).
The set $A$ is referred to as the \textbf{domain}, while $B$ is called the \textbf{codomain}. As a convention, $f: A \rightarrow B$ means that \emph{every} point in $A$ is sent somewhere; no point in $A$ is left unassigned.
On the other hand, we don't require that every $b \in B$ is an output realized by $f$. For example,
$\sin : \mathbb{R} \rightarrow \mathbb{R}$ is a function that is indeed defined for every real number but only takes values in $[-1,1]$,
so any value outside of that range is not realized as an output of sine. This leads us to define the $\textbf{range}$ of $f$ as 
\[ f(A) = \{f(x): x \in A\}.\]
Note that it also makes sense to define $f(A')$ in the exact same way for some $A' \subset A$. $f(A')$ is called the \textbf{image}
of $A'$ under $f$. So one could just as well say that the range of $f$ is the image of the domain $A$ under $f$.

The $\textbf{graph}$ of $f$, sometimes denoted $\Gamma(f)$, is 
\[\Gamma(f) = \{(x,f(x)): x \in A\} \subset A \times B.\]
Note that for a real-valued $f: \mathbb{R} \rightarrow \mathbb{R}$, the graph of $f$ coincides with the usual 
notion of the graph of a function (i.e.\ it's a squiggle that $f$ traces out in the plane).





% ------------------------------------------------------------------------------

\end{document}