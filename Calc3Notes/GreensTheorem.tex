\documentclass{article}

\usepackage{amsmath, amsthm, amssymb, amsfonts}
\usepackage{thmtools}
\usepackage{graphicx}
\usepackage{setspace}
\usepackage{geometry}
\usepackage{float}
\usepackage{hyperref}
\usepackage[utf8]{inputenc}
\usepackage[english]{babel}
\usepackage{framed}
\usepackage[dvipsnames]{xcolor}
\usepackage{tcolorbox}

\colorlet{LightGray}{White!90!Periwinkle}
\colorlet{LightOrange}{Orange!15}
\colorlet{LightGreen}{Green!15}

\newcommand{\HRule}[1]{\rule{\linewidth}{#1}}

\declaretheoremstyle[name=Theorem,]{thmsty}
\declaretheorem[style=thmsty,numberwithin=section]{theorem}
\tcolorboxenvironment{theorem}{colback=LightGray}

\declaretheoremstyle[name=Proposition,]{prosty}
\declaretheorem[style=prosty,numberlike=theorem]{proposition}
\tcolorboxenvironment{proposition}{colback=LightOrange}

\declaretheoremstyle[name=Principle,]{prcpsty}
\declaretheorem[style=prcpsty,numberlike=theorem]{principle}
\tcolorboxenvironment{principle}{colback=LightGreen}

\setstretch{1.2}
\geometry{
    textheight=9in,
    textwidth=5.5in,
    top=1in,
    headheight=12pt,
    headsep=25pt,
    footskip=30pt
}

% ------------------------------------------------------------------------------

\begin{document}

% ------------------------------------------------------------------------------
% Cover Page and ToC
% ------------------------------------------------------------------------------


% ------------------------------------------------------------------------------
Let us see how Green's theorem works in the case of a rectangular region.
Let $R=[a,b]\times[c,d]$ be a rectangular region, and let $C$ denote the boundary
curve, which is a rectangle, oriented counterclockwise. 
Let $F = \langle M,N \rangle$ be a smooth vector field in the plane, meaning
$M(x,y)$ and $N(x,y)$ are smooth real-valued functions. The integral of $F$ around $C$ is
\[ \oint_C \vec{F} \cdot d\vec{r} = \oint_C Mdx + Ndy = \oint_C Mdx + \oint_C Ndy. \]

Let us inspect $\oint_C M dx$. If we parametrize $C$ and evaluate this integral, 
we notice that the vertical segments contribute nothing, since $dx=0$ when we traverse a vertical segment.
So the integral reduces to
\[\oint_C M dx = \int_a^b M(x,c) dx - \int_a^b M(x,d)dx=\int_a^b \left( M(x,c)-M(x,d)\right)dx.\]
Now, for a fixed $x$, $M(x,y)$ is a function of $y$ which we can 
differentiate with respect to $y$, as $\partial M(x,y) / \partial y$. The fundamental theorem
of calculus tells us that
\[M(x,d) - M(x,c) = \int_c^d \frac{\partial M(x,y)}{\partial y}dy. \]
Substituting this back into our prior integral, noting the difference is sign, we obtain
\[\oint_C M dx = \int_a^b \int_c^d -\frac{\partial M(x,y)}{\partial y}dydx=\int_R -\frac{\partial M}{\partial y}dA.\]

The exact same work will show that 
\[\int_C N dy = \int_R \frac{\partial N}{\partial x} dA,\]
where this time there's no negative sign since the orientations agree now. 

% ------------------------------------------------------------------------------

\end{document}