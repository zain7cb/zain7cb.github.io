\documentclass{article}

\usepackage{amsmath, amsthm, amssymb, amsfonts}
\usepackage{thmtools}
\usepackage{graphicx}
\usepackage{setspace}
\usepackage{geometry}
\usepackage{float}
\usepackage{hyperref}
\usepackage[utf8]{inputenc}
\usepackage[english]{babel}
\usepackage{framed}
\usepackage[dvipsnames]{xcolor}
\usepackage{tcolorbox}
\usepackage{physics}
\DeclareMathOperator{\grd}{grad}
\colorlet{LightGray}{White!90!Periwinkle}
\colorlet{LightOrange}{Orange!15}
\colorlet{LightGreen}{Green!15}

\newcommand{\HRule}[1]{\rule{\linewidth}{#1}}

\declaretheoremstyle[name=Theorem,]{thmsty}
\declaretheorem[style=thmsty,numberwithin=section]{theorem}
\tcolorboxenvironment{theorem}{colback=LightGray}

\declaretheoremstyle[name=Proposition,]{prosty}
\declaretheorem[style=prosty,numberlike=theorem]{proposition}
\tcolorboxenvironment{proposition}{colback=LightOrange}

\declaretheoremstyle[name=Principle,]{prcpsty}
\declaretheorem[style=prcpsty,numberlike=theorem]{principle}
\tcolorboxenvironment{principle}{colback=LightGreen}

\setstretch{1.2}
\geometry{
    textheight=9in,
    textwidth=5.5in,
    top=1in,
    headheight=12pt,
    headsep=25pt,
    footskip=30pt
}

% ------------------------------------------------------------------------------

\begin{document}

% ------------------------------------------------------------------------------
% Cover Page and ToC
% ------------------------------------------------------------------------------
\section*{Critical Points}

A point $P$ is a \textbf{critical point} of $f$ if $\grd f (P) = O$. 
Equivalently, all the partial derivatives $D_i f$ are $0$ at $P$.

\textbf{Example.} 
Find the critical points of $f(x,y) = e^{-(x^2+y^2)}$. We take 
partial derivatives and set them to $0$ to find the critical points.

As in the single variable case, we can have a variety of 
behaviors at a critical point; we do not necessarily have
a local minimum or local maximum.

Let $f$ be defined on an open set $U$. A point $P$ is called a
\textbf{local maximum} of $f$ if, in some
neighborhood $N$ of $P$, we have 
\[f(X) \leq f(P)\]
for all $X \in N$.

The concept of local minimum is defined similarly. 

\textbf{Theorem.} Let $f$ be a differentiable function on $U$.
Let $P$ be a local maximum. Then $P$ is a critical point of $f$.

The proof of this amounts to reducing it to a one variable problem.
If $H$ is a nonzero vector, and $t$ is small enough, then 
$P + tH \in U$. Moreover, if $t$ is small enough, $P+tH$ will land
in the neighborhood mentioned in the definition, so that
\[f(P+tH) \leq f(P)\]
for all $t$ in an interval of the form $(-\delta, \delta)$, $\delta > 0$.
So $g(t) = f(P+tH)$ has a local maximum at $t=0$. Thus $g'(t)=0$. 
By the chain rule, 
\[\grd f (P) \cdot H = 0.\]
This is true for all $H$, so we must have $\grd f (P) = 0$. $\blacksquare$

A similar argument shows that local minima are also critical points 
of $f$.

\section*{Boundary, Interior, etc.}

An \textbf{open ball} of radius $r>0$ in $\mathbb{R}^n$ centered at $P$ is defined to be 
the set of all points $X$ such that $\norm{X-P} < r$. 

A \textbf{closed ball} is similarly defined except $\norm{X-P} \leq r$ (rather than strict inequality).

A subset $U \subseteq \mathbb{R}^n$ is \textbf{open} if at every point $P \in U$, there is a ball of some radius around $P$
contained entirely in $U$.

An \textbf{interior point} $P$ of a set $S \subseteq \mathbb{R}^n$ is one such there exists a ball
of some radius around $P$ contained entirely in $S$. Thus one could rephrase the definition of openness
as each point being an interior point.

A point $P$ (not necessarily in $S$) is called a \textbf{boundary point} of $S$ if every open ball around 
$P$ contains both a point in $S$ and a point not in $S$.

A set is \textbf{closed} if it contains all of its boundary points.

A set is \textbf{bounded} if one can fit the set inside a ball. Equivalently, $S$ is bounded
if there is some $b>0$ such that $\norm{X} \leq b$ for all $X\in S$.
\end{document}