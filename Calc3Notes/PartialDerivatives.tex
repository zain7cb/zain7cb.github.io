\documentclass{article}

\usepackage{amsmath, amsthm, amssymb, amsfonts}
\usepackage{thmtools}
\usepackage{graphicx}
\usepackage{setspace}
\usepackage{geometry}
\usepackage{float}
\usepackage{hyperref}
\usepackage[utf8]{inputenc}
\usepackage[english]{babel}
\usepackage{framed}
\usepackage[dvipsnames]{xcolor}
\usepackage{tcolorbox}
\usepackage{physics}
\DeclareMathOperator{\grd}{grad}
\colorlet{LightGray}{White!90!Periwinkle}
\colorlet{LightOrange}{Orange!15}
\colorlet{LightGreen}{Green!15}

\newcommand{\HRule}[1]{\rule{\linewidth}{#1}}

\declaretheoremstyle[name=Theorem,]{thmsty}
\declaretheorem[style=thmsty,numberwithin=section]{theorem}
\tcolorboxenvironment{theorem}{colback=LightGray}

\declaretheoremstyle[name=Proposition,]{prosty}
\declaretheorem[style=prosty,numberlike=theorem]{proposition}
\tcolorboxenvironment{proposition}{colback=LightOrange}

\declaretheoremstyle[name=Principle,]{prcpsty}
\declaretheorem[style=prcpsty,numberlike=theorem]{principle}
\tcolorboxenvironment{principle}{colback=LightGreen}

\setstretch{1.2}
\geometry{
    textheight=9in,
    textwidth=5.5in,
    top=1in,
    headheight=12pt,
    headsep=25pt,
    footskip=30pt
}

% ------------------------------------------------------------------------------

\begin{document}

% ------------------------------------------------------------------------------
% Cover Page and ToC
% ------------------------------------------------------------------------------


% ------------------------------------------------------------------------------


\section*{Functions of Several Variables}

Lang has a very specific definition of function. He requires that 
the output of $f$ is a number. The input can be any subset of $n$-space.


\textbf{Example.} $f: \mathbb{R}^2 \rightarrow \mathbb{R}$ defined by
$f(x,y)=\sqrt{x^2+y^2}$. We can interpret $f$ as a function that 
tells us our distance to the origin when we're standing at a point
$(x,y)$.

\textbf{Example.} $f: \mathbb{R}^3 \rightarrow \mathbb{R}$ defined 
by $f(x,y,z) = x^2 - \sin(xyz) + yz^3$.

The graph of a function on defined on $S\subset \mathbb{R}^2$ would have the form
\[ \{ (x,y,f(x,y)) : (x,y) \in S\}.\]
In this case, the graph sits in $\mathbb{R}^3$.

For a fixed number $c$, the equation $f(x,y)=c$ describes a curve
in $\mathbb{R}^2$. Such a curve is called a \textbf{level curve}.

\textbf{Question.} What do the level curves of $f(x,y) = x^2 + y^2$ look like? What about $f(x,y) = \sqrt{x^2+y^2}$.


If $f(x,y,z)$ is a function of three variables, the equation $f(x,y,z)=c$
describes a surface, called a \textbf{level surface}.

\textbf{Question.} What do the level surfaces of $f(x,y,z) = x^2 + y^2 + z^2$ look like?
What about $f(x,y,z) = 3x^2 + 2y^2 + z$?


\section*{Partial Derivatives}

First consider a function of two variables $f(x,y)$. If we hold one of the variables
fixed and allow the other to vary, we obtain a function of one variables, and we
can take the derivative as we did in Calc I:
\[\lim_{h \to 0} \frac{f(x+h,y)-f(x,y)}{h}.\]
This is the \textbf{partial derivative with respect to the first variable} or the 
\textbf{partial derivative with respect to x}. The second partial derivative would be 
\[\lim_{h \to 0} \frac{f(x,y+h)-f(x,y)}{h}.\]

Notations for this include $D_1 f, D_2 f$; $\frac{\partial f}{\partial x}, \frac{\partial f}{\partial y}$; $f_x, f_y$. 
And of course, we can extend these ideas to functions of $3$ or more variables.

\textbf{Example.} Let $f(x,y) = x^2 y^3$. To compute $\partial f / \partial x$, we treat $y$ as a constant
and differentiate as usual:
\[\frac{\partial f}{\partial x} = 2xy^3.\]
Similarly, 
\[\frac{\partial f}{\partial y} = 3x^2y^2.\]
% ------------------------------------------------------------------------------

Geometrically, for functions of two variables, taking a partial derivative corresponds to 
slicing the graph at $x=a$ or $y=a$ for a constant $a$ and then 
looking at the slope of the tangent.

Note that $D_i f$ is itself a function that we can evaluate at points.

\textbf{Example.} Let $f(x,y) = \sin(xy)$. Compute $D_2 f (1,\pi)$.
\[D_2 f(x,y) = \cos(xy)x.\]
So then 
\[D_2 f(1,\pi) = \cos(\pi)\cdot 1 = -1.\]

Notice that we can use vector notation and write the partial derivative with respect to $x_i$ as 
\[(D_i f)(X) = \lim_{h \to 0} \frac{f(X+hE_i)-f(X)}{h}.\]

The \textbf{gradient} of a function is the vector-valued function
\[\grd f(x,y) = \left( \frac{\partial f}{\partial x}, \frac{\partial f}{\partial y} \right).\]
One can easily generalize this definition to higher dimensions. 

\textbf{Example.} Let $f(x,y,z) = x^2 y \sin(yz)$. Find $\grd f(1,1,\pi)$.

We note that for functions $f,g$ and any constant $c$
\[\grd(f+g)=\grd f + \grd g,\ \grd(cf)=c \grd f.\]

\section*{Continuity}
(For this section, I'll be borrowing from both Lang and Courant)

Let $x=(x_1,\ldots,x_n)$ and $y=(y_1,\ldots,y_n)$ be points. The distance
between them is 
\[ d(x,y)=\norm{x-y}= \sqrt{(x_1-y_1)^2+\cdots+(x_n-y_n)^2} .\]
In the case where $n=1$ (i.e. when $x$ and $y$ are just numbers), we see
the distance is the absolute value of the difference $x-y$ (recall that $\sqrt{x^2}=|x|$).

If $a$ and $b$ are \emph{nonnegative} and $a\leq b$, then 
$\sqrt{a} \leq \sqrt{b}$. Conversely, if $\sqrt{a} \leq \sqrt{b}$, then $a\leq b$. In particular,
we have $|x|=\sqrt{x^2} \leq \sqrt{x^2 + y^2}$. In the same way, $|y| \leq \sqrt{x^2+y^2}$, and of course,
this works for more than just two variables.


Recall the idea of a function being continuous at $x_0 \in X$. Intuitively, this means that
as $x$ gets closer to $x_0$, $f(x)$ gets closer $f(x_0)$.

\textbf{Definition.} 
A function $f(x,y)$ defined on a region $R$ is \textbf{continuous} at a point $(x_0,y_0) \in R$ if, 
for every positive number $\epsilon$,  one can find a corresponding positive number $\delta_\epsilon$ such that
$d((x_0,y_0),(x,y)) < \delta_\epsilon$ (where $(x,y) \in R$) guarantees $d(f(x,y),f(x_0,y_0)) < \epsilon$. 

\textbf{Example.} 
Let's look at a single variable example first. Let $f(x) = 10x$. 
Let us show that $f$ is continuous at $0$. So let $x_0 = 0$. Then 
$f(x_0)=f(0)=0$. Now, somebody hands us some $\epsilon > 0$, and we 
need to respond by finding a $\delta > 0$ such that $x$ being within 
$\delta$ of $x_0$ guarantees that $f(x)$ is within $\epsilon$ of $f(x_0)$.
Now, 
\[|f(x)-f(x_0)| = |10x|=10|x|,\]
while
\[|x-x_0|=|x|.\]

Let $\delta = \epsilon/10$. Then if $d(x_0,x)=|x-x_0| < \delta$, we have
\[|f(x)-f(x_0)| = 10|x| = 10|x-x_0| < 10(\epsilon/10)=\epsilon.\]
This shows that $f$ is continuous at $x_0=0$. In this case, the same argument works at any $x_0$, not just $0$.

A function of several variables can have discontinuties of a more
complicated type. 

\textbf{Example.} Let $f(x,y) = \frac{2xy}{x^2+y^2}$, $f(0,0)=0$. 
If we approach the origin along the $x$-axis (i.e. along $y=0$), we have
$f(x,0)=0$. Similarly, $f(0,y)=0$. On the other hand, if we approach
the origin along the line $y=x$, we have $f(x,x) = 2x^2/(2x^2)=1$.
Thus $f$ cannot be continuous at $(0,0)$.

\textbf{Definition.} We say that the \textbf{limit} of $f(x,y)$ as $(x,y)$ approaches $(x_0,y_0)$ equals $L$ if, 
for every positive number $\epsilon$, one can find a corresponding positive number $\delta_\epsilon$ such that
$0 < d((x_0,y_0),(x,y)) < \delta_\epsilon$ guarantees $d(f(x,y),L) < \epsilon$. One writes
\[\lim_{(x,y)\to (x_0,y_0)} f(x,y) = L.\]

\section*{Differentiability}
In initial attempt to define differentiability of functions
of several variables, one might be tempted to write down something like
\[\frac{f(X+H)-f(X)}{H}.\]
This doesn't make sense, though, because we have no notion 
of dividing by vectors.

Let us look back to the single variable case first. The derivative
in this case was defined to be
\[f'(x) = \lim_{h \to 0} \frac{f(x+h) - f(x)}{h}.\]
Let 
\[\varphi(h) = \frac{f(x+h) - f(x)}{h} - f'(x).\]
$\varphi(h)$ is not defined for $h=0$, but 
\[\lim_{h \to 0} \varphi(h) = 0.\]
We can go ahead and just set $\varphi(0)=0$.
We can also write 
\[f(x+h) - f(x) = f'(x)h + h\varphi(h),\]
and this is true for $h=0$ now as well. We will now do a silly thing:
let $g(h) = \varphi(h)$ for $h>0$ and $g(h) = -\varphi(h)$ for $h<0$. This is simply
so that we now have
\[f(x+h) - f(x) = f'(x)h + |h|g(h)\]
where $\lim_{h\to 0} g(h)=0$.

Conversely, suppose that there exists some number $a$ 
and a function $g(h)$ 
with $\lim_{h \to 0} g(h) = 0$ such that
\[f(x+h) - f(x) = ah + |h|g(h).\]
Then when $h=0$, we have
\[\frac{f(x+h)-f(x)}{h} = a + \frac{h}{|h|} g(h).\]
As $h \to 0$, the rightmost term goes to $0$ and we see that $f$
is differentiable. In fact, $f'(x) = a$.


This dicussion is all to say that we could very well take this
to be the definition of differentiability. In words, differentiability
at a point $x_0$ is equivalent to a function having a ``good'' linear approximation.
Here, ``good'' means that the difference between the approximation and $f$
shrinks faster than $|h|$. This is maybe more easily seen if we write
\[f(x+h) - \left( f(x) + ah \right) = |h|g(h)\]
and then dividing by $|h|$.

It is this definition that will be easier to adapt to the higher dimension 
situation. We will start with two variables. Let $X=(x,y)$, $H=(h,k)$,
so that $X+H=(x+h,y+k)$ and 
\[f(X+H)-f(X) = f(x+h,y+k) - f(x,y).\]

\textbf{Definition.} $f$ is differentiable at $X$ if the partial derivatives 
exist and if there exists a function $g$ (defined for small $H$) 
with $\lim_{H \to O} g(H) = 0$ such that
\[f(X+H) - f(X) = \grd(f)\cdot H + \norm{H}g(H).\]

\end{document}