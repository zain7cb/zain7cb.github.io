\documentclass{article}

\usepackage{amsmath, amsthm, amssymb, amsfonts}
\usepackage{thmtools}
\usepackage{graphicx}
\usepackage{setspace}
\usepackage{geometry}
\usepackage{float}
\usepackage{hyperref}
\usepackage[utf8]{inputenc}
\usepackage[english]{babel}
\usepackage{framed}
\usepackage[dvipsnames]{xcolor}
\usepackage{tcolorbox}
\usepackage{physics}
\DeclareMathOperator{\grd}{grad}
\colorlet{LightGray}{White!90!Periwinkle}
\colorlet{LightOrange}{Orange!15}
\colorlet{LightGreen}{Green!15}

\newcommand{\HRule}[1]{\rule{\linewidth}{#1}}

\declaretheoremstyle[name=Theorem,]{thmsty}
\declaretheorem[style=thmsty,numberwithin=section]{theorem}
\tcolorboxenvironment{theorem}{colback=LightGray}

\declaretheoremstyle[name=Proposition,]{prosty}
\declaretheorem[style=prosty,numberlike=theorem]{proposition}
\tcolorboxenvironment{proposition}{colback=LightOrange}

\declaretheoremstyle[name=Principle,]{prcpsty}
\declaretheorem[style=prcpsty,numberlike=theorem]{principle}
\tcolorboxenvironment{principle}{colback=LightGreen}

\setstretch{1.2}
\geometry{
    textheight=9in,
    textwidth=5.5in,
    top=1in,
    headheight=12pt,
    headsep=25pt,
    footskip=30pt
}

% ------------------------------------------------------------------------------

\begin{document}

% ------------------------------------------------------------------------------
% Cover Page and ToC
% ------------------------------------------------------------------------------


% ------------------------------------------------------------------------------


\section*{Functions of Several Variables}

Lang has a very specific definition of function. He requires that 
the output of $f$ is a number. The input can be any subset of $n$-space.


\textbf{Example.} $f: \mathbb{R}^2 \rightarrow \mathbb{R}$ defined by
$f(x,y)=\sqrt{x^2+y^2}$. We can interpret $f$ as a function that 
tells us our distance to the origin when we're standing at a point
$(x,y)$.

\textbf{Example.} $f: \mathbb{R}^3 \rightarrow \mathbb{R}$ defined 
by $f(x,y,z) = x^2 - \sin(xyz) + yz^3$.

The graph of a function on defined on $S\subset \mathbb{R}^2$ would have the form
\[ \{ (x,y,f(x,y)) : (x,y) \in S\}.\]
In this case, the graph sits in $\mathbb{R}^3$.

For a fixed number $c$, the equation $f(x,y)=c$ describes a curve
in $\mathbb{R}^2$. Such a curve is called a \textbf{level curve}.

\textbf{Question.} What do the level curves of $f(x,y) = x^2 + y^2$ look like? What about $f(x,y) = \sqrt{x^2+y^2}$.


If $f(x,y,z)$ is a function of three variables, the equation $f(x,y,z)=c$
describes a surface, called a \textbf{level surface}.

\textbf{Question.} What do the level surfaces of $f(x,y,z) = x^2 + y^2 + z^2$ look like?
What about $f(x,y,z) = 3x^2 + 2y^2 + z$?


\section*{Partial Derivatives}

First consider a function of two variables $f(x,y)$. If we hold one of the variables
fixed and allow the other to vary, we obtain a function of one variables, and we
can take the derivative as we did in Calc I:
\[\lim_{h \to 0} \frac{f(x+h,y)-f(x,y)}{h}.\]
This is the \textbf{partial derivative with respect to the first variable} or the 
\textbf{partial derivative with respect to x}. The second partial derivative would be 
\[\lim_{h \to 0} \frac{f(x,y+h)-f(x,y)}{h}.\]

Notations for this include $D_1 f, D_2 f$; $\frac{\partial f}{\partial x}, \frac{\partial f}{\partial y}$; $f_x, f_y$. 
And of course, we can extend these ideas to functions of $3$ or more variables.

\textbf{Example.} Let $f(x,y) = x^2 y^3$. To compute $\partial f / \partial x$, we treat $y$ as a constant
and differentiate as usual:
\[\frac{\partial f}{\partial x} = 2xy^3.\]
Similarly, 
\[\frac{\partial f}{\partial y} = 3x^2y^2.\]
% ------------------------------------------------------------------------------

Geometrically, for functions of two variables, taking a partial derivative corresponds to 
slicing the graph at $x=a$ or $y=a$ for a constant $a$ and then 
looking at the slope of the tangent.

Note that $D_i f$ is itself a function that we can evaluate at points.

\textbf{Example.} Let $f(x,y) = \sin(xy)$. Compute $D_2 f (1,\pi)$.
\[D_2 f(x,y) = \cos(xy)x.\]
So then 
\[D_2 f(1,\pi) = \cos(\pi)\cdot 1 = -1.\]

Notice that we can use vector notation and write the partial derivative with respect to $x_i$ as 
\[(D_i f)(X) = \lim_{h \to 0} \frac{f(X+hE_i)-f(X)}{h}.\]

The \textbf{gradient} of a function is the vector-valued function
\[\grd f(x,y) = \left( \frac{\partial f}{\partial x}, \frac{\partial f}{\partial y} \right).\]
One can easily generalize this definition to higher dimensions. 

\textbf{Example.} Let $f(x,y,z) = x^2 y \sin(yz)$. Find $\grd f(1,1,\pi)$.

We note that for functions $f,g$ and any constant $c$
\[\grd(f+g)=\grd f + \grd g,\ \grd(cf)=c \grd f.\]

\section*{Differentiability and Gradient}


\end{document}