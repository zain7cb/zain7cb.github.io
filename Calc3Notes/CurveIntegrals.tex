\documentclass{article}

\usepackage{amsmath, amsthm, amssymb, amsfonts}
\usepackage{thmtools}
\usepackage{graphicx}
\usepackage{setspace}
\usepackage{geometry}
\usepackage{float}
\usepackage{hyperref}
\usepackage[utf8]{inputenc}
\usepackage[english]{babel}
\usepackage{framed}
\usepackage[dvipsnames]{xcolor}
\usepackage{tcolorbox}
\usepackage{physics}
\DeclareMathOperator{\grd}{grad}
\colorlet{LightGray}{White!90!Periwinkle}
\colorlet{LightOrange}{Orange!15}
\colorlet{LightGreen}{Green!15}

\newcommand{\HRule}[1]{\rule{\linewidth}{#1}}

\declaretheoremstyle[name=Theorem,]{thmsty}
\declaretheorem[style=thmsty,numberwithin=section]{theorem}
%\tcolorboxenvironment{theorem}{colback=LightGray}

\declaretheoremstyle[name=Proposition,]{prosty}
\declaretheorem[style=prosty,numberlike=theorem]{proposition}
\tcolorboxenvironment{proposition}{colback=LightOrange}

\declaretheoremstyle[name=Principle,]{prcpsty}
\declaretheorem[style=prcpsty,numberlike=theorem]{principle}
\tcolorboxenvironment{principle}{colback=LightGreen}

\setstretch{1.2}
\geometry{
    textheight=9in,
    textwidth=5.5in,
    top=1in,
    headheight=12pt,
    headsep=25pt,
    footskip=30pt
}

% ------------------------------------------------------------------------------

\begin{document}

% ------------------------------------------------------------------------------

\section*{Curve Integrals}

Recall (or accept) from physics that the work 
(which has the same units as energy) done by a constant force $F$ over a distance $D$ is 
$W=FD$. This describes the case of the force pointing in the direction
of motion. A slightly more general equation is $W = F \cdot D$, where
$F$ is the force vector and $D$ is the displacement vector (imagine pushing a box). But this 
equation still assumes a straight-line displacement and constant force in a fixed direction.
What if our trajectory is a curve $C(t)$ and the force is a vector quantity $F(X)$ that depends
on position?

If one zooms in close enough on a continuous vector field, it looks
constant, and similarly a curve will look like a straight line segment.
The work done by the force on a small time interval $(t, t + \Delta t)$ can then be approximated as 
\[F(C(t))\cdot (C(t+\Delta t)-C(t)).\]
We can rewrite this as 
\[F(C(t))\cdot \frac{C(t+\Delta t) - C(t)}{\Delta t}\Delta t.\]
If we add up these small bits of work and let $\Delta t \to 0$, we end
up with an integral.

Thus we define the \textbf{integral of $F$ along $C$} from time $a$ to time $b$ as 
\[\int_C F = \int_a^b F(C(t))\cdot \frac{dC}{dt} dt.\]

\textbf{Example.} $F(x,y) = (x^2y, y^3)$. Find the integral along the straight line
from $(0,0)$ to $(1,1)$.

We take $C(t) = (t,t)$, $0 \leq t \leq 1$. $C'(t) = (1,1)$. Then 
\[F(C(t)) = (t^3,t^3).\]
Our integral is then
\[\int_0^1 (t^3,t^3)\cdot (1,1)dt = \int_0^1 2t^3 dt = 1/2.\]

In $2$-space, if we write $F = (f,g)$, $C(t) = (x(t),y(t))$, then the curve integral
can be expressed
\[\int_C F = \int_C fdx + gdy.\]
Symbolically, the expression $fdx + gdy = (f,g)\cdot (dx,dy)$. So one can write
\[\int_C F = \int_a^b \left[ f(x(t),y(t))\frac{dx}{dt} + g(x(t),y(t))\frac{dy}{dt}\right]dt.\]
\end{document}
