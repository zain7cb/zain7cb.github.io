\documentclass{article}

\usepackage{amsmath, amsthm, amssymb, amsfonts}
\usepackage{thmtools}
\usepackage{graphicx}
\usepackage{setspace}
\usepackage{geometry}
\usepackage{float}
\usepackage{hyperref}
\usepackage[utf8]{inputenc}
\usepackage[english]{babel}
\usepackage{framed}
\usepackage[dvipsnames]{xcolor}
\usepackage{tcolorbox}
\usepackage{physics}
\DeclareMathOperator{\grd}{grad}
\colorlet{LightGray}{White!90!Periwinkle}
\colorlet{LightOrange}{Orange!15}
\colorlet{LightGreen}{Green!15}

\newcommand{\HRule}[1]{\rule{\linewidth}{#1}}

\declaretheoremstyle[name=Theorem,]{thmsty}
\declaretheorem[style=thmsty,numberwithin=section]{theorem}
\tcolorboxenvironment{theorem}{colback=LightGray}

\declaretheoremstyle[name=Proposition,]{prosty}
\declaretheorem[style=prosty,numberlike=theorem]{proposition}
\tcolorboxenvironment{proposition}{colback=LightOrange}

\declaretheoremstyle[name=Principle,]{prcpsty}
\declaretheorem[style=prcpsty,numberlike=theorem]{principle}
\tcolorboxenvironment{principle}{colback=LightGreen}

\setstretch{1.2}
\geometry{
    textheight=9in,
    textwidth=5.5in,
    top=1in,
    headheight=12pt,
    headsep=25pt,
    footskip=30pt
}

% ------------------------------------------------------------------------------

\begin{document}

% ------------------------------------------------------------------------------
% Cover Page and ToC
% ------------------------------------------------------------------------------
\section*{The Chain Rule}

\textbf{Example.} Let $f(x,y) = e^x \sin(xy)$. One can imagine
this describes the temperature of the plane at each point $(x,y)$.
Now imagine a bug moving in the plane with parametrization
$C(t) = (t^2, t^3)$. Then the temperature the bug is feeling at time $t$
is 
\[f(C(t))=e^{t^2}\sin(t^5).\]

Of course, we could compute the derivative of this directly,
but there's another way.

\textbf{Chain Rule.} Let $f$ be differentiable on an open set $U$
and let $C(t)$ be a differentiable curve contained in $U$. Then 
\[\frac{d}{dt} f\circ C(t) = (\grd f)(C(t)) \cdot C'(t).\]

Suppose we're in the two-variable case and 
$C(t) = (x(t),y(t))$. We could rewrite the chain rule as
\[\frac{d}{dt} f\circ C(t) = \frac{\partial f}{\partial x} \frac{dx}{dt} + \frac{\partial f}{\partial y} \frac{dy}{dt}\]
where the partial derivatives are of course evaluated at $(x(t),y(t))$.

\textbf{Example.} Let $f(x,y,z) = x^2 y z$ and 
$C(t) = (x(t),y(t),z(t)) = (e^t, t, t^2)$.
Then 
\begin{align*}
(f\circ C)'(t) &= (D_1 f) x'(t) + (D_2 f) y'(t) + (D_3 f) z'(t)\\
&= 2xyz e^t + x^2 z  + x^2 y (2t)\\
&= 2e^{2t} t^3 + e^{2t} t^2 + 2e^{2t} t^2.
\end{align*}

There are situations where we need only use the standard single variable chain rule.
\textbf{Example.}
Let $f(x,y,z)=\sin(x^2-3yz+xz)$.
Then 
\[\frac{\partial f}{\partial x} = \cos(x^2-3yz+xz)(2x + z).\]
% ------------------------------------------------------------------------------
% ------------------------------------------------------------------------------

\section*{Tangent Plane}
Let $f(x,y,z)$ be a function on $\mathbb{R}^3$. Imagine that $f$ models the 
temperature at each point of the space and that we have
a bug moving along a curve $B(t) = (x(t),y(t),z(t))$ in space. 
Assume the bug started at a point with a comfortable temperature $k$
and so decides to stick to points with temperature $k$. That is,
the bug is moving on the level surface
\[f(x,y,z)=k.\]
That is, we have for all $t$ that
\[f(B(t))=k.\]
Applying chain rule, we have
\[(\grd f)(B(t))\cdot B'(t) = 0.\]
So the gradient of $f$ is perpendicular to the path of the bug
at every point.

In general, if we fix a point $P$ on a level surface $f(x,y,z)=k$ and look at all
differentiable curves passing through $P$ at, say, $t=0$, the above computation shows that all such curves
will be be perpendicular to $\grd{f}(P)$ at $t=0$ (see the following figure from Lang).
\begin{figure}[h]
    \centering
    \includegraphics[scale = 0.5]{tangent1.PNG}
\end{figure}
Thus, in a very real sense, $\grd f(P)$ is perpendicular to the surface $f(x,y,z)=k$ itself.
This leads to the following definition.

\textbf{Definition.} The \textbf{tangent plane} to $f(X)=k$ at $P$ is the plane through $P$,
perpendicular to $\grd f(P)$.

\textbf{Example.} Find the tangent plane to $x^2+y^2+z^2=3$ at the point $(1,1,1)$. 

Note that this a level surface of the function $f(x,y,z)=x^2+y^2+z^3$ (corresponding to $f=3$). 
So our normal vector is $N = (2x,2y,2z)|_{(1,1,1)} = (2,2,2)$. The plane equation is then
\[(2,2,2)\cdot(x-1,y-1,z-1)=0,\]
or
\[x+y+z=3.\]

\end{document}