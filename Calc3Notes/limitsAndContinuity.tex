\documentclass{article}

\usepackage{amsmath, amsthm, amssymb, amsfonts}
\usepackage{thmtools}
\usepackage{graphicx}
\usepackage{setspace}
\usepackage{geometry}
\usepackage{float}
\usepackage{hyperref}
\usepackage[utf8]{inputenc}
\usepackage[english]{babel}
\usepackage{framed}
\usepackage[dvipsnames]{xcolor}
\usepackage{tcolorbox}
\usepackage{physics}

\colorlet{LightGray}{White!90!Periwinkle}
\colorlet{LightOrange}{Orange!15}
\colorlet{LightGreen}{Green!15}

\newcommand{\HRule}[1]{\rule{\linewidth}{#1}}

\declaretheoremstyle[name=Theorem,]{thmsty}
\declaretheorem[style=thmsty,numberwithin=section]{theorem}
\tcolorboxenvironment{theorem}{colback=LightGray}

\declaretheoremstyle[name=Proposition,]{prosty}
\declaretheorem[style=prosty,numberlike=theorem]{proposition}
\tcolorboxenvironment{proposition}{colback=LightOrange}

\declaretheoremstyle[name=Principle,]{prcpsty}
\declaretheorem[style=prcpsty,numberlike=theorem]{principle}
\tcolorboxenvironment{principle}{colback=LightGreen}

\declaretheoremstyle[name=Definition,]{prcpsty}
\declaretheorem[style=prcpsty,numberlike=theorem]{definition}
\tcolorboxenvironment{definition}{colback=LightGreen}

\setstretch{1.2}
\geometry{
    textheight=9in,
    textwidth=5.5in,
    top=1in,
    headheight=12pt,
    headsep=25pt,
    footskip=30pt
}

% ------------------------------------------------------------------------------

\begin{document}

% ------------------------------------------------------------------------------
% Cover Page and ToC
% ------------------------------------------------------------------------------


% ------------------------------------------------------------------------------
\section*{Inequalities, absolute value, distance, etc.}
Let $x=(x_1,\ldots,x_n)$ and $y=(y_1,\ldots,y_n)$ be points. The distance
between them is 
\[ d(x,y)=\norm{x-y}= \sqrt{(x_1-y_1)^2+\cdots+(x_n-y_n)^2} .\]
In the case where $n=1$ (i.e. when $x$ and $y$ are just numbers), we see
the distance is the absolute value of the difference $x-y$ (recall that $\sqrt{x^2}=|x|$).

If $a$ and $b$ are \emph{nonnegative} and $a\leq b$, then 
$\sqrt{a} \leq \sqrt{b}$. Conversely, if $\sqrt{a} \leq \sqrt{b}$, then $a\leq b$. In particular,
we have $|x|=\sqrt{x^2} \leq \sqrt{x^2 + y^2}$. In the same way, $|y| \leq \sqrt{x^2+y^2}$, and of course,
this works for more than just two variables.

\section*{Limits}
Let $f: X \to Y$ be a function taking points in $X$ to points in $Y$. In Calc I, $X$ and $Y$ were usually both $\mathbb{R}$. Now we allow $X$ to be $\mathbb{R}^2$ or $\mathbb{R}^3$.  

Recall the idea of a function having a limit at a point $x_0 \in X$. Intuitively, this means there is
some value $L$ such that when $x$ gets closer to $x_0$, $f(x)$ gets closer to this value $L$. Before we get lost 
in technicalities, let's look at some examples of evaluating limits of multivariable functions. 

\textbf{Example.} Compute 
\[\lim_{(x,y) \to (1,\pi)} f(x,y) = \frac{x}{y} + \cos(xy).\]
In this case, $f(x,y)$ is continuous at $(1,\pi)$, so evaluating this limit amounts to plugging in
$(1,\pi)$, so the limit is $1/\pi + \cos(\pi) = 1/\pi -1$.

\begin{definition}
    We say that the \textbf{limit} of $f(x,y)$ as $(x,y)$ approaches $(x_0,y_0)$ equals $L$ if, 
    for every positive number $\epsilon$, one can find a corresponding positive number $\delta_\epsilon$ such that
    $d((x_0,y_0),(x,y)) < \delta_\epsilon$ guarantees $d(f(x,y),L) < \epsilon$. One writes
    \[\lim_{(x,y)\to (x_0,y_0)} f(x,y) = L.\]
\end{definition}



Perhaps this nonsense is best understood through some examples. Consider the function 
\[f(x,y)=\frac{x^2 y^2}{x^2+y^2}.\]
This function is not defined at $(0,0)$, but does it still have a limit as $(x,y) \to (0,0)$? 

One would like to get away from the madness of using deltas and epsilons to demonstrate that a function
has a limit at a given point. Fortunately, by establishing a few properties and formulas, one can then wield
these properties and formulas to compute a wide variety of limits.

\begin{proposition} (Limit properties)


Suppose for the below that $\lim_{(x,y) \to (x_0,y_0)} f(x,y) = L$ and $\lim_{(x,y) \to (x_0,y_0)} g(x,y) = M$. Here,
$f$ and $g$ are real-valued.
    \begin{itemize}
        \item $\lim_{(x,y) \to (x_0,y_0)} x = x_0$ 
        \item $\lim_{(x,y) \to (x_0,y_0)} y = y_0$
        \item $\lim_{(x,y) \to (x_0,y_0)} f(x,y) \pm g(x,y) = M \pm L$
        \item $\lim_{(x,y) \to (x_0,y_0)} f(x,y)g(x,y) = LM$
        \item $\lim_{(x,y) \to (x_0,y_0)} f(x,y)/g(x,y) = L/M$ provided $M\neq 0$.
    \end{itemize}
\end{proposition}

Showing that these properties hold is a little technical, so we'll take them on faith.
% ------------------------------------------------------------------------------

\end{document}