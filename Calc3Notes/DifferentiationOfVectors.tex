\documentclass{article}

\usepackage{amsmath, amsthm, amssymb, amsfonts}
\usepackage{thmtools}
\usepackage{graphicx}
\usepackage{setspace}
\usepackage{geometry}
\usepackage{float}
\usepackage{hyperref}
\usepackage[utf8]{inputenc}
\usepackage[english]{babel}
\usepackage{framed}
\usepackage[dvipsnames]{xcolor}
\usepackage{tcolorbox}

\colorlet{LightGray}{White!90!Periwinkle}
\colorlet{LightOrange}{Orange!15}
\colorlet{LightGreen}{Green!15}

\newcommand{\HRule}[1]{\rule{\linewidth}{#1}}

\declaretheoremstyle[name=Theorem,]{thmsty}
\declaretheorem[style=thmsty,numberwithin=section]{theorem}
\tcolorboxenvironment{theorem}{colback=LightGray}

\declaretheoremstyle[name=Proposition,]{prosty}
\declaretheorem[style=prosty,numberlike=theorem]{proposition}
\tcolorboxenvironment{proposition}{colback=LightOrange}

\declaretheoremstyle[name=Principle,]{prcpsty}
\declaretheorem[style=prcpsty,numberlike=theorem]{principle}
\tcolorboxenvironment{principle}{colback=LightGreen}

\setstretch{1.2}
\geometry{
    textheight=9in,
    textwidth=5.5in,
    top=1in,
    headheight=12pt,
    headsep=25pt,
    footskip=30pt
}

% ------------------------------------------------------------------------------

\begin{document}

% ------------------------------------------------------------------------------
% Cover Page and ToC
% ------------------------------------------------------------------------------


% ------------------------------------------------------------------------------

\section*{Differentiation}

Imagine a bug that moves with constant speed on a circular path of 
radius $r$ around the origin.
The angle of the bug's position vector with the $+x$ axis can
be written as 
\[\theta = \omega t + a.\]
Assume $a=0$, so that the bug is on the $+x$ axis at time $0$.
Then the position vector of the bug is 
\[X(t) = (r \cos (\omega t), r \sin(\omega t)).\]

Now imagine the bug lives in $\mathbb{R}^3$ with
\[X(t) = (\cos(t), \sin(t), t).\]
This lifts the circular path into a helix.

In general, a \textbf{parametrized curve} $X : I \to \mathbb{R}^n$ is a vector-valued function
that maps points from an interval $I$ into $n$-space. In the examples above,
$I$ is the entire real line $\mathbb{R}$ (which we consider to be an interval). 
We can write $X(t)$ as its individual coordinate functions
\[X(t) = (x_1(t), \ldots, x_n(t)).\]
Just as with ordinary real-valued function, we can take derivatives by 
looking at the limit
\[\lim_{h \to 0} \frac{X(t+h)-X(t)}{h}.\]
Writing out components, this is simply
\[\lim_{h \to 0} \frac{(x_1(t+h)-x_1(t), \ldots, x_n(t+h)-x_n(t))}{h}.\]
If the individual components are all differentiable, we obtain a
new vector-valued function
\[X'(t) = ()\]



% ------------------------------------------------------------------------------

\end{document}