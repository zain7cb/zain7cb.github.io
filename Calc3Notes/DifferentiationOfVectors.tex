\documentclass{article}

\usepackage{amsmath, amsthm, amssymb, amsfonts}
\usepackage{thmtools}
\usepackage{graphicx}
\usepackage{setspace}
\usepackage{geometry}
\usepackage{float}
\usepackage{hyperref}
\usepackage[utf8]{inputenc}
\usepackage[english]{babel}
\usepackage{framed}
\usepackage[dvipsnames]{xcolor}
\usepackage{tcolorbox}
\usepackage{physics}
\colorlet{LightGray}{White!90!Periwinkle}
\colorlet{LightOrange}{Orange!15}
\colorlet{LightGreen}{Green!15}

\newcommand{\HRule}[1]{\rule{\linewidth}{#1}}

\declaretheoremstyle[name=Theorem,]{thmsty}
\declaretheorem[style=thmsty,numberwithin=section]{theorem}
\tcolorboxenvironment{theorem}{colback=LightGray}

\declaretheoremstyle[name=Proposition,]{prosty}
\declaretheorem[style=prosty,numberlike=theorem]{proposition}
\tcolorboxenvironment{proposition}{colback=LightOrange}

\declaretheoremstyle[name=Principle,]{prcpsty}
\declaretheorem[style=prcpsty,numberlike=theorem]{principle}
\tcolorboxenvironment{principle}{colback=LightGreen}

\setstretch{1.2}
\geometry{
    textheight=9in,
    textwidth=5.5in,
    top=1in,
    headheight=12pt,
    headsep=25pt,
    footskip=30pt
}

% ------------------------------------------------------------------------------

\begin{document}

% ------------------------------------------------------------------------------
% Cover Page and ToC
% ------------------------------------------------------------------------------


% ------------------------------------------------------------------------------

\section*{Differentiation}

Imagine a bug that moves with constant speed on a circular path of 
radius $r$ around the origin.
The angle of the bug's position vector with the $+x$ axis can
be written as 
\[\theta = \omega t + a.\]
Assume $a=0$, so that the bug is on the $+x$ axis at time $0$.
Then the position vector of the bug is 
\[X(t) = (r \cos (\omega t), r \sin(\omega t)).\]

Now imagine the bug lives in $\mathbb{R}^3$ with
\[X(t) = (\cos(t), \sin(t), t).\]
This lifts the circular path into a helix.

In general, a \textbf{parametrized curve} $X : I \to \mathbb{R}^n$ is a vector-valued function
that maps points from an interval $I$ into $n$-space. In the examples above,
$I$ is the entire real line $\mathbb{R}$ (which we consider to be an interval). 
We can write $X(t)$ as its individual coordinate functions
\[X(t) = (x_1(t), \ldots, x_n(t)).\]
Just as with ordinary real-valued function, we can take derivatives by 
looking at the limit
\[\lim_{h \to 0} \frac{X(t+h)-X(t)}{h}.\]
Here, dividing by $h$ really means scaling the vector by $1/h$. Writing out components, this is simply
\[\lim_{h \to 0} \frac{(x_1(t+h)-x_1(t), \ldots, x_n(t+h)-x_n(t))}{h}.\]
If the individual components are all differentiable, we obtain a
new vector-valued function
\[X'(t) = (x_1'(t), \ldots, x_n'(t)).\]
$X'(t)$ is called the \textbf{derivative} or \textbf{velocity} of $X(t)$. 

So for the example $X(t) = (\cos(t), \sin(t), t)$, we have
\[X'(t) = (-\sin(t), \cos(t), 1).\]

The velocity is parallel to the direction of instantaneous motion.


\textbf{Example.} 
Find a parametric equation of the tangent line to the curve $X(t) = (\sin t, \cos t)$
at $t=\pi/3$. 

We need two pieces of information: a point on the line, and a direction vector of the line.
These are supplied by $X(\pi/3)$ and $X'(\pi/3)$ respectively. The tangent line $L(t)$ can thus be 
written 
\begin{align*}
    L(s) |_{t=\pi/3} &= X(\pi/3) + sX'(\pi/3)  \\
    &= \left( \frac{\sqrt{3}}{2} + \frac{1}{2}s,\ \frac{1}{2} - \frac{\sqrt{3}}{2}s \right).
\end{align*}
We used the parameter $s$ for the line to avoid confusion with the already defined $X(t)$ above.

The \textbf{speed} of the curve $X(t)$, denoted $v(t)$, is defined to be 
\[v(t) = \norm{X'(t)}.\]

\textbf{acceleration} is the second derivative $X''(t)$.

We note also that differentiation is linear, meaning
\[\frac{d}{dt} \left( X(t) + Y(t) \right) = X'(t) + Y'(t)\]
and 
\[\frac{d}{dt} cX(t) = cX'(t).\]

We also have a product rule:
\[\frac{d}{dt} X(t) \cdot Y(t) = X'(t)\cdot Y(t) + X(t) \cdot Y'(t).\]
This follows from applying the ordinary product rule. If $X(t) = (x_1(t), x_2(t))$ and
$Y(t) = (y_1(t), y_2(t))$, then
\begin{align*}
    \frac{d}{dt} X(t) \cdot Y(t) &= \frac{d}{dt} (x_1y_1 + x_2y_2)\\
    &= x_1'y_1 + x_1y_1' + x_2'y_2 + x_2y_2'\\
    &= x_1'y_1 + x_2'y_2 + x_1y_1' + x_2y_2'\\
    &= X'(t)\cdot Y(t) + X(t) \cdot Y'(t).
\end{align*}
Of course, this same argument works in dimensions higher than 2.

Lang uses the notation $X(t)^2$ for $X(t)\cdot X(t)=\norm{X(t)}^2$. Using this, the above formula
has as a particular case
\[\frac{d}{dt} X(t)^2 = 2 X(t)\cdot X'(t). \]
% ------------------------------------------------------------------------------
\section*{Length of Curves}
If we integrate the speed $v(t)$ of $X(t)$ from time $t=a$ to $t=b$,
we obtain the distance or length traveled by $X(t)$ during the time interval $[a,b]$:
\[\text{length} = \int_a^b v(t)dt.\]

\textbf{Example.} Let $X(t) = (\cos(t), \sin(t))$ describe a particle.
What distance does $X(t)$ traverse from $t=0$ to $t=1$?

We have $X'(t)=(-\sin(t), \cos(t))$. Then $v(t) = \norm{X'(t)} 
= \sqrt{(-\sin(t))^2 + \cos^2(t)} = 1$. So the distance $D$ is 
\[D = \int_0^1 1 dt = 1.\]

Note that distance and displacement are not the same thing. In the example above,
if we consider the distance traveled from $t=0$ to $t=2\pi$, the particle
travels a distance of $2\pi$, but the net displacement is $0$ since it
ends up where it started.

Suppose $X(t) = (x_1(t), x_2(t))$. Then the length integral can be written as
\[\int_a^b \sqrt{\left( \frac{dx_1}{dt} \right)^2+ \left( \frac{dx_2}{dt} \right)^2}dt.\]
This might seem familiar. In fact, consider now a real-valued function $f(x)$. We can parametrize 
the graph of $f$ from $x=a$ to $x=b$ as 
\[X(t) = (t,f(t)),\ a\leq t \leq b.\]
Slotting this into the integral above gives
\[\int_a^b \sqrt{1+f'(t)^2}dt,\]
which is the arclength formula you may have seen in Calc II.

\end{document}