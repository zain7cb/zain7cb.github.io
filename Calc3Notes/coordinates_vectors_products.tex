\documentclass{article}

\usepackage{amsmath, amsthm, amssymb, amsfonts}
\usepackage{thmtools}
\usepackage{graphicx}
\usepackage{setspace}
\usepackage{geometry}
\usepackage{float}
\usepackage{hyperref}
\usepackage[utf8]{inputenc}
\usepackage[english]{babel}
\usepackage{framed}
\usepackage[dvipsnames]{xcolor}
\usepackage{tcolorbox}

\colorlet{LightGray}{White!90!Periwinkle}
\colorlet{LightOrange}{Orange!15}
\colorlet{LightGreen}{Green!15}

\newcommand{\HRule}[1]{\rule{\linewidth}{#1}}

\declaretheoremstyle[name=Theorem,]{thmsty}
\declaretheorem[style=thmsty,numberwithin=section]{theorem}
\tcolorboxenvironment{theorem}{colback=LightGray}

\declaretheoremstyle[name=Proposition,]{prosty}
\declaretheorem[style=prosty,numberlike=theorem]{proposition}
\tcolorboxenvironment{proposition}{colback=LightOrange}

\declaretheoremstyle[name=Principle,]{prcpsty}
\declaretheorem[style=prcpsty,numberlike=theorem]{principle}
\tcolorboxenvironment{principle}{colback=LightGreen}

\setstretch{1.2}
\geometry{
    textheight=9in,
    textwidth=5.5in,
    top=1in,
    headheight=12pt,
    headsep=25pt,
    footskip=30pt
}

% ------------------------------------------------------------------------------

\begin{document}

% ------------------------------------------------------------------------------
% Cover Page and ToC
% ------------------------------------------------------------------------------


% ------------------------------------------------------------------------------



\section*{Points in 3-space and beyond}

In the same way that we can specify a point in the plane with 
two numbers, we can specify a point in space with three numbers
$(x,y,z)$. In general, in $n$-space ($\mathbb{R}^n$), we can specify
a point with a list of $n$ numbers $(x_1,\ldots,x_n)$.

Given two points in $\mathbb{R}^3$, we can define addition on them 
by adding corresponding coordinates:
\[(a_1,a_2,a_3)+(b_1,b_2,b_3):=(a_1+b_1, a_2+b_2, a_3+b_3).\]
In general,
\[(a_1,\ldots,a_n)+(b_1,\ldots,b_n):=(a_1+b_1,\ldots,a_n+b_n).\]

\textbf{Example.}
Let $A=(2,3)$, $B=(-1,1)$. Then $A+B=(1,4)$.
The figure looks like a parallelogram.

\textbf{Example.}
Let $A=(3,1)$, $B=(1,2)$. Then $A+B=(4,3)$.
We obtain a parallelogram again. This is always the case.
Starting from the origin $O=(0,0)$, we obtain $B$ by moving $1$
unit right and then $2$ units up. We get $A+B$ by first moving 
$3$ to the right, then $1$ up, and then repeating the same movement 
we did from the origin to $B$. In other words, the segment connecting
$O$ to $B$ and the one connecting $A$ to $A+B$ are equal length and parallel.
Similarly, the segments from $O$ to $A$ and $B$ to $B+A = A+B$ will also be equal length and parallel.

We have some not-so-surprising properties of point addition
\begin{itemize}
    \item $(A+B)+C=A+(B+C)$
    \item $A + B = B + A$
    \item $O + A = A + O = A$
    \item $A + (-A) = O$
\end{itemize}
where $O=(0,\ldots,0)$ and $-A = (-a_1, \ldots, -a_n)$.
We note that $A \mapsto -A$ corresponds to reflection about the origin.

We can also multiply (or \emph{scale}) a point $A=(a_1,\ldots,a_n)$ by a number $c$, yielding a point
\[cA = (ca_1, \ldots, ca_n).\]
For example, if $A=(2,-1,5)$ and $c=7$, then $cA=(14,-7,35)$.
We again have some easy properties:
\begin{itemize}
    \item $c(A+B)=cA + cB$
    \item $(c_1 + c_2)A = c_1 A + c_2 A$
    \item $(c_1 c_2)A = c_1(c_2 A)$.
\end{itemize}
We should comment on the geometric meaning of scaling by a number $c$. 
Let $A=(1,2)$ and $c=3$. Then $cA=(3,6)$. We see that the effect of multiplying by $3$ is to
stretch the point $A$ away from the origin by a factor of $3$. If we set $c=1/2$, this 
shrinks $A$ in towards the origin. If we draw a segment from the origin to $A$, in the former case,
scaling by $c=3$ multiplies the length by $3$, and scaling by $c=1/2$ cuts the length in half.

\section*{Vectors}
The discussion above leads us naturally to vectors. Given two points $A$ and $B$,
we can define a \textbf{located vector} as an ordered pair of points $(A,B)$, which is more
often written $\overrightarrow{AB}$. We think of this as an arrow connecting $A$ and $B$,
pointing towards $B$. Two located vectors $\overrightarrow{AB}$ and $\overrightarrow{CD}$
are said to be \textbf{equivalent} if $B-A=D-C$. We always have that
$\overrightarrow{AB}$ is equivalent to $\overrightarrow{O(B-A)}$. This is actually the unique
vector starting at the origin that is equivalent to $\overrightarrow{AB}$.

$\overrightarrow{AB}$ and $\overrightarrow{PQ}$ are said to be \textbf{parallel} if 
for some $c\neq 0$, we have $A-B = c(Q-P)$. If $c>0$ we say the vectors have the \emph{same direction},
and if $c<0$, we say they have the \emph{opposite direction}. 

$\overrightarrow{AB}$ and $\overrightarrow{PQ}$ are said to be \textbf{perpendicular} if 
$B - A$ and $Q - P$ are perpendicular in the usual geometric sense.

\section*{The Dot Product}




% ------------------------------------------------------------------------------
\end{document}