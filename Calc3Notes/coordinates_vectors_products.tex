\documentclass{article}

\usepackage{amsmath, amsthm, amssymb, amsfonts}
\usepackage{thmtools}
\usepackage{graphicx}
\usepackage{setspace}
\usepackage{geometry}
\usepackage{float}
\usepackage{hyperref}
\usepackage[utf8]{inputenc}
\usepackage[english]{babel}
\usepackage{framed}
\usepackage[dvipsnames]{xcolor}
\usepackage{tcolorbox}
\usepackage{physics}

\colorlet{LightGray}{White!90!Periwinkle}
\colorlet{LightOrange}{Orange!15}
\colorlet{LightGreen}{Green!15}

\newcommand{\HRule}[1]{\rule{\linewidth}{#1}}

\declaretheoremstyle[name=Theorem,]{thmsty}
\declaretheorem[style=thmsty,numberwithin=section]{theorem}
\tcolorboxenvironment{theorem}{colback=LightGray}

\declaretheoremstyle[name=Proposition,]{prosty}
\declaretheorem[style=prosty,numberlike=theorem]{proposition}
\tcolorboxenvironment{proposition}{colback=LightOrange}

\declaretheoremstyle[name=Principle,]{prcpsty}
\declaretheorem[style=prcpsty,numberlike=theorem]{principle}
\tcolorboxenvironment{principle}{colback=LightGreen}

\setstretch{1.2}
\geometry{
    textheight=9in,
    textwidth=5.5in,
    top=1in,
    headheight=12pt,
    headsep=25pt,
    footskip=30pt
}

% ------------------------------------------------------------------------------

\begin{document}

% ------------------------------------------------------------------------------
% Cover Page and ToC
% ------------------------------------------------------------------------------


% ------------------------------------------------------------------------------



\section*{Points in 2-space, 3-space, and Beyond}

In the same way that we can specify a point in the plane with 
two numbers, we can specify a point in space with three numbers
$(x,y,z)$. In general, in $n$-space ($\mathbb{R}^n$), we can specify
a point with a list of $n$ numbers $(x_1,\ldots,x_n)$.

Given two points in $\mathbb{R}^3$, we can define addition on them 
by adding corresponding coordinates:
\[(a_1,a_2,a_3)+(b_1,b_2,b_3):=(a_1+b_1, a_2+b_2, a_3+b_3).\]
In general,
\[(a_1,\ldots,a_n)+(b_1,\ldots,b_n):=(a_1+b_1,\ldots,a_n+b_n).\]

\textbf{Example.}
Let $A=(2,3)$, $B=(-1,1)$. Then $A+B=(1,4)$.
The figure looks like a parallelogram.

\textbf{Example.}
Let $A=(3,1)$, $B=(1,2)$. Then $A+B=(4,3)$.
We obtain a parallelogram again. This is always the case.
Starting from the origin $O=(0,0)$, we obtain $B$ by moving $1$
unit right and then $2$ units up. We get $A+B$ by first moving 
$3$ to the right, then $1$ up, and then repeating the same movement 
we did from the origin to $B$. In other words, the segment connecting
$O$ to $B$ and the one connecting $A$ to $A+B$ are equal length and parallel.
Similarly, the segments from $O$ to $A$ and $B$ to $B+A = A+B$ will also be equal length and parallel.

We have some not-so-surprising properties of point addition
\begin{itemize}
    \item $(A+B)+C=A+(B+C)$
    \item $A + B = B + A$
    \item $O + A = A + O = A$
    \item $A + (-A) = O$
\end{itemize}
where $O=(0,\ldots,0)$ and $-A = (-a_1, \ldots, -a_n)$.
We note that $A \mapsto -A$ corresponds to reflection about the origin.

We can also multiply (or \emph{scale}) a point $A=(a_1,\ldots,a_n)$ by a number $c$, yielding a point
\[cA = (ca_1, \ldots, ca_n).\]
For example, if $A=(2,-1,5)$ and $c=7$, then $cA=(14,-7,35)$.
We again have some easy properties:
\begin{itemize}
    \item $c(A+B)=cA + cB$
    \item $(c_1 + c_2)A = c_1 A + c_2 A$
    \item $(c_1 c_2)A = c_1(c_2 A)$.
\end{itemize}
We should comment on the geometric meaning of scaling by a number $c$. 
Let $A=(1,2)$ and $c=3$. Then $cA=(3,6)$. We see that the effect of multiplying by $3$ is to
stretch the point $A$ away from the origin by a factor of $3$. If we set $c=1/2$, this 
shrinks $A$ in towards the origin. If we draw a segment from the origin to $A$, in the former case,
scaling by $c=3$ multiplies the length by $3$, and scaling by $c=1/2$ cuts the length in half.

\section*{Vectors}
The discussion above leads us naturally to vectors. Given two points $A$ and $B$,
we can define a \textbf{located vector} as an ordered pair of points $(A,B)$, which is more
often written $\overrightarrow{AB}$. We think of this as an arrow connecting $A$ and $B$,
pointing towards $B$. Two located vectors $\overrightarrow{AB}$ and $\overrightarrow{CD}$
are said to be \textbf{equivalent} if $B-A=D-C$. We always have that
$\overrightarrow{AB}$ is equivalent to $\overrightarrow{O(B-A)}$. This is actually the unique
vector starting at the origin that is equivalent to $\overrightarrow{AB}$.

$\overrightarrow{AB}$ and $\overrightarrow{PQ}$ are said to be \textbf{parallel} if 
for some $c\neq 0$, we have $A-B = c(Q-P)$. If $c>0$ we say the vectors have the \emph{same direction},
and if $c<0$, we say they have the \emph{opposite direction}. 

$\overrightarrow{AB}$ and $\overrightarrow{PQ}$ are said to be \textbf{perpendicular} if 
$B - A$ and $Q - P$ are perpendicular in the usual geometric sense.

A located vector starting from the origin is completely determined by its
endpoint. So an $n$-tuple will be called either a point or a \textbf{vector}
depending on the context and interpretation. 

\section*{The Dot Product}

If $\vec{x} = (x_1,x_2, \ldots, x_n)$ and $\vec{y} = (y_1, y_2, \ldots, y_n)$,
their \textbf{dot product} is 
\[\vec{x} \cdot \vec{y} = x_1 y_1 + x_2 y_2 + \cdots + x_n y_n.\]
Note 

Some useful properties are

\begin{enumerate}
    \item $A \cdot B = B \cdot A$
    \item $A \cdot (B + C) = A \cdot B + A \cdot C = (B+C) \cdot A$
    \item If $x$ is a number, $(xA) \cdot B = x(A \cdot B)$, $A \cdot (xB) = x(A \cdot B)$
    \item If $A=O$, then $A \cdot A = 0$. Otherwise, $A \cdot A > 0$.
\end{enumerate}

Two vectors $A$ and $B$ are said to be \textbf{perpendicular} or \textbf{orthogonal} if
$A \cdot B = 0$. For the plane and $\mathbb{R}^3$, we will see that this
definition agrees with our previous and more geometric definition of perpendicular.

The \textbf{norm} or \textbf{magnitude} (or \textbf{length}) $\norm{A}$ of a vector $A=(a_1, \ldots, a_n)$ is 
\[\norm{A} = \sqrt{A \cdot A} = \sqrt{a_1^2 + \cdots + a_n^2}.\]

Note that $\norm{-A} = \norm{A}$. More generally, for any number $c$, 
we have $\norm{cA} = |c|\norm{A}$. For two points $A,B$, the \textbf{distance}
between them is 
\[\norm{A-B}= \sqrt{(A-B)\cdot(A-B)}.\]

A vector $E$ is a \textbf{unit vector} if $\norm{E}=1$. Dividing a nonzero vector by its norm always yields
a unit vector, since 
\[\norm{\frac{A}{\norm{A}}} = \frac{1}{\norm{A}} \norm{A} = 1.\]

Two nonzero vectors $A$ and $B$ have the \textbf{same direction} if there is some $c>0$ such that
$cA = B$. So, for instance, $A/\norm{A}$ is a unit vector in the same direction as $A$.

\section*{Perpendicularity, Angle Between Vectors}

We have two notions of ``perpendicular'' floating around. One says $A$ and $B$ are perpendicular if $A \cdot B = 0$.
The other is the more familiar notion of $A$ and $B$ forming a right angle. Suppose that $A$ and $B$ lie in the plane.
We can convince ourselves that $A$ and $B$ form a right angle precisely when
\[\norm{A-B} = \norm{A+B}.\]

\includegraphics[scale = 0.5]{perpendicular.PNG}

If we accept this, then the equivalence of our two definitions of perpendicularity will follow from 
\[\norm{A+B}=\norm{A-B} \iff A \cdot B = 0.\]

To prove this, observe that
\begin{align*}
    \norm{A+B}=\norm{A-B}&\iff \norm{A + B}^2 = \norm{A - B}^2\\
    &\iff A\cdot A + 2 A \cdot B + B \cdot B = A \cdot A - 2 A \cdot B + B \cdot B\\
    &\iff 4 A \cdot B = 0\\
    &\iff A \cdot B = 0.
\end{align*}

Suppose again that we have two nonzero vectors $A$ and $B$ in the plane, located at the origin.
If we move along the line through $\overrightarrow{OB}$, there will be some point $P$ on this line such
that $\overrightarrow{PA}$ is perpendicular to $\overrightarrow{OB}$. Then $P = c B$ for some number $c$.
Then we have $(A - P)\cdot B = (A - cB)\cdot B = 0$, which is to say
\[A \cdot B - c B \cdot B = 0,\]
so that
\[c = \frac{A \cdot B}{B \cdot B}.\]

Conversely, we see that
\[\left( A - \frac{A \cdot  B}{B \cdot B} B \right) \cdot B = A\cdot B - A \cdot B = 0.\]
Thus, this is the unique number $c$ that makes $A - cB$ perpendicular to $B$. This number $c$ is called
the \textbf{component} of $A$ along $B$. If we do a little plane geometry, we see that
\[\cos \theta = \frac{c \norm{B}}{\norm{A}},\]
which can be rewritten as 
\[\norm{A} \norm{B} \cos \theta = A \cdot B.\]

The \textbf{projection} of $A$ onto $B$ is 
\[\frac{A \cdot B}{B \cdot B} B.\]
Note that if $D = cB$, then 
\begin{align*}
    \frac{A \cdot D}{D \cdot D} D &= \frac{A \cdot cB}{cB \cdot cB} cB\\
    &= \frac{A \cdot B}{B \cdot B} B,
\end{align*}
so the projection of $A$ onto $B$ doesn't strictly depend on $B$; projecting onto any multiple of
$B$ will yield the same vector (the component, however, \emph{does} up to a sign).

Note also that if $B$ is a unit vector, the component simplifies to $A \cdot B$, and the projection of 
$A$ onto $B$ simplifies to $(A \cdot B) B$.

\section*{Parametric Lines}
Given a direction vector $A$ and a point $P$, the \textbf{parametric line}
in the direction of $A$ passing through $P$ is given by
\[X(t) = P + tA,\]
where $t$ ranges in $\mathbb{R}$. One can think of this as 
the position $X$ of a particle or bug as it travels with
the passing of time $t$. $X(t)$ is sometimes called the
\textbf{position vector} of the particle/bug. The position vector
is a vector located at the origin, terminating at the position of 
the bug. The figure below illustrates how the parametrization works:
we start $P$, and as $t$ varies, we shift $P$ by a multiple of $A$. As $t$ ranges over all of
$\mathbb{R}$, this traces out a line.
\begin{figure}[h]
    \centering
    \includegraphics[scale = 0.5]{paramLine.PNG}
\end{figure}
    
Given two points $P$ and $Q$, we can parametrize the line segment 
between them as 
\[X(t) = P + t(Q-P),\ 0 \leq t \leq 1.\]
\begin{figure}[h]
    \centering
    \includegraphics[scale = 0.5]{segment.PNG}
\end{figure}

\section*{Planes}

A plane $M$ in $\mathbb{R}^3$ is determined by two pieces of data: 
a point $P$ lying on the plane, and a vector $N$ perpendicular
to the plane. If we walk from $P$ to some other point $X$ also on the plane,
we must have that $X-P$ is perpendicular to $N$, otherwise $X$ won't lie
on the plane $M$. So the plane is the set of points $X$ satisfying
\[N \cdot (X-P) = 0.\]
\begin{figure}[h]
    \centering
    \includegraphics[scale = 0.5]{planes.PNG}
\end{figure}

Note that this gives a nice interpretation of the equation for a line
in the plane $ax+by=c$. $(a,b)$ is a normal vector to the line!
If $c=0$, the equation becomes $(a,b)\cdot (x,y) = 0$, so we have
a line through the origin, consisting of all vectors (located at the origin)
perpendicular to $(a,b)$. Changing the value of $c$ yields a family
of parallel lines.

\begin{figure}[h]
    \centering
    \includegraphics[scale = 0.5]{distancetoplane.PNG}
\end{figure}

Suppose that we have a plane passing through $P$ and perpendicular to $N$, and let 
$Q$ be some point not on the plane. How can we compute the (smallest) distance of $Q$ to the plane?
The smallest distance corresponds to the length of the segment formed by drawing a perpendicular to
the plane from the point $Q$. We can obtain this length by projecting $Q-P$ onto $N$ and taking the norm:

\[\text{length} = \left|  (Q-P)\cdot \frac{N}{\norm{N}} \right|\]

\section*{The Cross Product}

If $A=(a_1,a_2,a_3)$ and $B=(b_1,b_2,b_3)$ are vectors in $\mathbb{R}^3$,
their cross product $A \times B$ is the determinant
\[ \begin{vmatrix}
    E_1 & E_2 & E_3\\
    a_1 & a_2 & a_3\\
    b_1 & b_2 & b_3
    \end{vmatrix} .\]

This is really more of mnemonic device than an actual definition, since
the determinant is defined only for matrices with numerical entries. 

$A \times B$ is perpendicular to both $A$ and $B$. We also have 
\emph{anticommutativity}, meaning $B \times A = -(A \times B)$.

One can verify that $\norm{A \times B}^2 = \norm{A}^2 \norm{B}^2 - (A \cdot B)^2$.

Using our geometric formula for the dot product, we have
\begin{align*}
    \norm{A \times B}^2 &= \norm{A}^2 \norm{B}^2 - \norm{A}^2 \norm{B}^2 \cos^2(\theta)\\
    &= \norm{A}^2 \norm{B}^2 (1 - \cos^2(\theta))\\
    &= \norm{A}^2 \norm{B}^2 \sin^2(\theta).
\end{align*}
Taking square roots, we have
\[\norm{A\times B} = \norm{A} \norm{B} |\sin(\theta)|. \]
So the magnitude of the cross product is the area of parallelogram
spanned by $A$ and $B$.

\begin{figure}[h]
    \centering
    \includegraphics[scale = 0.5]{crossprod.PNG}
\end{figure}

% ------------------------------------------------------------------------------
\end{document}