\documentclass{article}

\usepackage{amsmath, amsthm, amssymb, amsfonts}
\usepackage{thmtools}
\usepackage{graphicx}
\usepackage{setspace}
\usepackage{geometry}
\usepackage{float}
\usepackage{hyperref}
\usepackage[utf8]{inputenc}
\usepackage[english]{babel}
\usepackage{framed}
\usepackage[dvipsnames]{xcolor}
\usepackage{tcolorbox}
\usepackage{physics}

\DeclareMathOperator{\grd}{grad}
\colorlet{LightGray}{White!90!Periwinkle}
\colorlet{LightOrange}{Orange!15}
\colorlet{LightGreen}{Green!15}

\newcommand{\HRule}[1]{\rule{\linewidth}{#1}}

\declaretheoremstyle[name=Theorem,]{thmsty}
\declaretheorem[style=thmsty,numberwithin=section]{theorem}
\tcolorboxenvironment{theorem}{colback=LightGray}

\declaretheoremstyle[name=Proposition,]{prosty}
\declaretheorem[style=prosty,numberlike=theorem]{proposition}
\tcolorboxenvironment{proposition}{colback=LightOrange}

\declaretheoremstyle[name=Principle,]{prcpsty}
\declaretheorem[style=prcpsty,numberlike=theorem]{principle}
\tcolorboxenvironment{principle}{colback=LightGreen}

\setstretch{1.2}
\geometry{
    textheight=9in,
    textwidth=5.5in,
    top=1in,
    headheight=12pt,
    headsep=25pt,
    footskip=30pt
}

% ------------------------------------------------------------------------------

\begin{document}


% ------------------------'

\textbf{HW 6.}

\textbf{Exercise 1.} 
Let $F(X) = r^n X$, where $r=\norm{X}$, the distance to the origin. 
Find a potential function for $F$. (Hint: recall that for a function 
$f = g(r)$, the gradient is 
\[\grd f(X) = \frac{g'(r)}{r}X.)\]

\textbf{Exercise 2.} 
Let $r = \sqrt{x^2+y^2}$. Compute the integral of \[F = \frac{1}{r}X\] around the circle
of radius $2$ centered at the origin, going counterclockwise.

\textbf{Exercise 3.} 
Let \[G(x,y) = \left( \frac{-y}{x^2+y^2}, \frac{x}{x^2+y^2}\right).\]
Compute the integral of $G$ along the following curves:
\begin{itemize}
    \item $x^2+y^2 = 2$ from $(1,1)$ to $(-\sqrt{2},0)$.
    \item Counterclockwise around the entire circle $x^2+y^2 = r^2$ where $r$ is some fixed radius.
\end{itemize}

\textbf{Exercise 4.} Integrate the field $F = (x^2y^2, xy^2)$ counterclockwise around
the closed path formed by the parts of the line $x=1$ and the 
parabola $x=y^2$.

\textbf{Exercise 5.} Let $r = \sqrt{x^2+y^2}$ and 
\[F(x,y) = \left( \frac{x}{r^3}, \frac{y}{r^3} \right).\]
Find the integral of $F$ along $C(t) = (e^t \cos t, e^t \sin t)$
from $(1,0)$ to $(e^{2\pi},0)$. (Hint: do you really want to 
evaluate this directly?)

\textbf{Exercise 6.} Find potentials for the following vector fields.
\begin{itemize}
    \item $(y\sin(z), x\sin(z), xy\cos(z))$
    \item $(z^2, 2y, 2xz)$
\end{itemize}

\end{document}