\documentclass{article}

\usepackage{amsmath, amsthm, amssymb, amsfonts}
\usepackage{thmtools}
\usepackage{graphicx}
\usepackage{setspace}
\usepackage{geometry}
\usepackage{float}
\usepackage{hyperref}
\usepackage[utf8]{inputenc}
\usepackage[english]{babel}
\usepackage{framed}
\usepackage[dvipsnames]{xcolor}
\usepackage{tcolorbox}

\colorlet{LightGray}{White!90!Periwinkle}
\colorlet{LightOrange}{Orange!15}
\colorlet{LightGreen}{Green!15}

\newcommand{\HRule}[1]{\rule{\linewidth}{#1}}

\declaretheoremstyle[name=Theorem,]{thmsty}
\declaretheorem[style=thmsty,numberwithin=section]{theorem}
\tcolorboxenvironment{theorem}{colback=LightGray}

\declaretheoremstyle[name=Proposition,]{prosty}
\declaretheorem[style=prosty,numberlike=theorem]{proposition}
\tcolorboxenvironment{proposition}{colback=LightOrange}

\declaretheoremstyle[name=Principle,]{prcpsty}
\declaretheorem[style=prcpsty,numberlike=theorem]{principle}
\tcolorboxenvironment{principle}{colback=LightGreen}

\setstretch{1.2}
\geometry{
    textheight=9in,
    textwidth=5.5in,
    top=1in,
    headheight=12pt,
    headsep=25pt,
    footskip=30pt
}

% ------------------------------------------------------------------------------

\begin{document}


% ------------------------------------------------------------------------------

\textbf{Exercise 1.}
 Let $A = (1,2)$, $B = (3,1)$. Draw the points $A+B$, $A+2B$, $A+3B$, $A-B$, $A-2B$, and $A - 3B$
 on a sheet of graph paper (or a reasonably drawn set of axes).

\textbf{Exercise 2.} Which of the following pairs of vectors are perpendicular?
\begin{itemize}
    \item $(1,-1,1), (2,1,5)$ 
    \item $(1,-1,1), (2,3,1)$ 
    \item $(-5,2,7), (3,-1,2)$
    \item $(\pi,2,1), (2,-\pi,0)$
\end{itemize}

\textbf{Exercise 3.}
Suppose $A = (a_1,a_2,a_3)$ is perpendicular to every vector $X$. Show that
$A$ is the zero vector.

\textbf{Exercise 4.}  Determine the interior angles of the triangle whose vertices are 
$(2, -1, 1)$, $(1, - 3, - 5)$, and $(3, -4, -4)$.

\textbf{Exercise 5.} Let $A_1, \ldots, A_r$ be vectors which are mutually perpendicular
(i.e. $A_i \cdot A_j = 0$ whenever $i \neq j$). Suppose $c_1, \ldots, c_r$ are numbers such that
\[c_1 A_1 + \cdots + c_r A_r = 0.\]
Show that we must have $c_i = 0$ for each $i=1,\ldots, r$.

\textbf{Exercise 6.} Let $P = (1,3, -1)$ and $Q = (-4,5,2)$. 
Determine the coordinates of the following points
\begin{itemize}
    \item The midpoint of the line segment between $P$ and $Q$
    \item The point on this line segment that is two thirds of the way from $P$ to $Q$.
\end{itemize}


% ------------------------------------------------------------------------------

\end{document}