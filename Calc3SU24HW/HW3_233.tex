\documentclass{article}

\usepackage{amsmath, amsthm, amssymb, amsfonts}
\usepackage{thmtools}
\usepackage{graphicx}
\usepackage{setspace}
\usepackage{geometry}
\usepackage{float}
\usepackage{hyperref}
\usepackage[utf8]{inputenc}
\usepackage[english]{babel}
\usepackage{framed}
\usepackage[dvipsnames]{xcolor}
\usepackage{tcolorbox}
\usepackage{physics}

\DeclareMathOperator{\grd}{grad}
\colorlet{LightGray}{White!90!Periwinkle}
\colorlet{LightOrange}{Orange!15}
\colorlet{LightGreen}{Green!15}

\newcommand{\HRule}[1]{\rule{\linewidth}{#1}}

\declaretheoremstyle[name=Theorem,]{thmsty}
\declaretheorem[style=thmsty,numberwithin=section]{theorem}
\tcolorboxenvironment{theorem}{colback=LightGray}

\declaretheoremstyle[name=Proposition,]{prosty}
\declaretheorem[style=prosty,numberlike=theorem]{proposition}
\tcolorboxenvironment{proposition}{colback=LightOrange}

\declaretheoremstyle[name=Principle,]{prcpsty}
\declaretheorem[style=prcpsty,numberlike=theorem]{principle}
\tcolorboxenvironment{principle}{colback=LightGreen}

\setstretch{1.2}
\geometry{
    textheight=9in,
    textwidth=5.5in,
    top=1in,
    headheight=12pt,
    headsep=25pt,
    footskip=30pt
}

% ------------------------------------------------------------------------------

\begin{document}


% ------------------------------------------------------------------------------

\textbf{HW 3}

\textbf{Exercise 1.}
Sketch the level curves for the following functions (just pick a few values for $f$).
\begin{itemize}
    \item $f(x,y)=xy$.
    \item $f(x,y)=(x-1)(y-2)$ (think of the relation this has with the previous function).
    \item $f(x,y)=2x-3y$.
\end{itemize}

\textbf{Exercise 2.} Find the partial derivatives $\partial f / \partial x$, $\partial f/ \partial y$,
and $\partial f / \partial z$ (where it applies) for the following functions.
\begin{itemize}
    \item $f(x,y,z) = e^{xyz}$.
    \item $f(x,y,z) = \sin(xy) + \cos(z)$.
\end{itemize}

\textbf{Exercise 3.} Find the gradient of $f(x,y,z) = e^{3x+y} \sin(5z)$ at $(0,0,\pi/6)$.

\textbf{Exercise 4.} Let $A=(a_1,a_2,a_3)$ and let $f$ be the function on $\mathbb{R}^3$ defined by
$f(X) = A\cdot X$ (since we're in $\mathbb{R}^3$, one can write $X=(x,y,z)$). What is $\grd f$?

\textbf{Exercise 5.} Let $f:\mathbb{R}^3 \to \mathbb{R}$ be defined on an open set $U$. Let $X \in U$. Suppose that we have
a vector $A$ and a function $g(H)$ defined for all $H$ in a neighborhood of $O$ (i.e. for small $H$)
such that
\[f(X+H) - f(X) = A \cdot H + \norm{H}g(H).\]
Show that $A = \grd f$. (Hint: use special values of $H$, such $hE_i$, where $h\in \mathbb{R}$ is small and 
$E_i$ is the $i$-th standard basis vector.) (Note: the point of this exercise is to show that you actually don't need
to assume that the partial derivatives exist in the definition of differentiability; the above
condition already ensures they exist, and that they in fact form the components of this vector $A$.)

\textbf{Exercise 6.} 
Let $f(x,y) = e^{9x+2y}$ and $g(x,y) = \sin(4x+y)$. Suppose $C(t)$ is a differentiable curve with
$C(0)=(0,0)$. Given:
\[\frac{d}{dt} f(C(t)) \bigg\vert_{t=0}=2,\ \frac{d}{dt} g(C(t)) \bigg\vert_{t=0}=1.\]
Find $C'(0)$.
(This problem is essentially saying the following: if $f$ and $g$ are appropriately chosen functions,
and I tell you how the values of $f$ and $g$ are changing as $C(t)$ goes through a point at time $t$,
then you work out the direction $C(t)$ is headed at time $t$.)
\end{document}